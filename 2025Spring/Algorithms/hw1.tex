%Example of use of oxmathproblems latex class for problem sheets
\documentclass{oxmathproblems}

%(un)comment this line to enable/disable output of any solutions in the file
%\printanswers

%define the page header/title info
\course{Algorithm Design and Analysis}
\sheettitle{Assignment 1 \\ Deadline: March 23, 2025} %can leave out if no title per sheet


\begin{document}

\begin{questions}
\miquestion[24]
Asymptotic notations.
\begin{parts}
    \part[24] In each of the following situations, indicate whether $f=O(g)$, or $f=\Omega(g)$, or both (in which case $f=\Theta(g)$). Justify your answer.
    \begin{enumerate}
        \item $f(n)=(n+1)!$ and $g(n)=n!$ \\
         \textbf{$f=\Omega(g)$ because $\frac{f(n)}{g(n)}=(n+1)$ is unbounded.}
        \item $f(n)=2^{n+1}$ and $g(n)=2^n$\\
        \textbf{$f=\Theta(g)$ because $\frac{f(n)}{g(n)}=2$ is a constant.}
        \item $f(n)=2^n$ and $g(n)=3^n$\\
        \textbf{$f=O(g)$ because $\frac{f(n)}{g(n)}=(\frac{2}{3})^n$ converges to 0.}
        \item $f(n)=n^{1/2}$ and $g(n)=5^{\log_2n}$\\
        \textbf{$f=O(g)$ because $\log_25>1/2$, so that $\frac{f(n)}{g(n)}=\frac{n^{1/2}}{n^{\log_25}}$ converges to 0.}
        \item $f(n)=100n+\log n$ and $g(n)=n+(\log n)^2$\\
        \textbf{$f=\Theta(g)$ because $\frac{f(n)}{g(n)}=100$ is a constant.}
        \item $f(n)=(\log n)^{\log n}$ and $g(n)=n/\log n$\\
        \textbf{$f=\Omega(g)$ because $\frac{f(n)}{g(n)}=(n+1)$ is unbounded.}
        \item $f(n)=(\log n)^{\log n}$ and $g(n)=2^{(\log_2n)^2}$\\
        \textbf{$f=O(g)$ because $\frac{f(n)}{g(n)}=\frac{(\log n)^{\log n}}{2^{(\log_2n)^2}}$ converges to 0.}
        \item $f(n)=\sum_{i=1}^ni^k$ and $g(n)=n^{k+1}$\\
        \begin{itemize}
            \item if k > -1, $f=\Theta(g)$ because $\frac{f(n)}{g(n)}=\frac{\sum_{i=1}^ni^k}{n^{k+1}}$ converges to $\frac{1}{k+1}.$
            \item if k <= -1, $f=O(g)$ because $\frac{f(n)}{g(n)}=\frac{\sum_{i=1}^ni^k}{n^{k+1}}$ is unbounded.
        \end{itemize}
    \end{enumerate}
    \part[Not for credit, just for fun: 0] Suppose $f:\mathbb{Z}^+\to\mathbb{R}^+$ and $g:\mathbb{Z}^+\to\mathbb{R}^+$ are increasing functions. Is it always true that we have either $f(n)=O(g(n))$ or $f(n)=\Omega(g(n))$?\\
    \textbf{I think the answer is no. Suppose $f(n)=2^n$, while n=2k and $f(n)=2^{n-1}+1$, while n=2k+1; $g(n)=2^{n-1}+1$, while n=2k and $g(n)=2^n$, while n=2k+1. Apparently, $f(n)=O(g(n))$, n=2k, while $f(n)=\Omega(g(n))$, n=2k+1, which means we have neither at $\mathbb{Z}^+$.}
\end{parts}

\miquestion[25]
Given an array $A[1\cdots n]$ of integers, a pair of indices $(i,j)$ is an \emph{inversion} if $i<j$ and $A[i]>A[j]$.
Design an algorithm that counts the number of inversions in $O(n\log n)$ time.\\
\textbf{Suppose p is the number of inversions.\\
    Devide the array A into two halves(let's say M,N with length m,n) with indices $(i,j)$.\\
    For i from 0 to m-1,j from 0 to n-1,repeat the following steps:\\
    1:Record $\min(Mi,Nj)$ in A'\\
    2:If Mi < Nj, then p=p+j and i=i+1; If Mi > Nj, then j=j+1 (a bit counterintuitive but practical)\\
    3:if i > n, break;or if j >m, then p=p+(n-i)*j and break.\\
    4:Record the rest to A'.\\
    Apparently, T(n)=2T(n/2)+O(2n), which means $T(n)=O(n\log n)$.}\\
\miquestion[25]
Given an array of $n$ integers $x_1,x_2,...,x_n$, there are queries of the following form: given an integer $1\leq k\leq n$, you need to return the $k$-th smallest integers in the array. Obviously, if we use $O(n\log n)$ time to preprocess the array by sorting it, we can answer each query in $O(1)$ time. In the class, we see that each query can be answered in $O(n)$ time without any preprocessing (the Median-of-the-Medians algorithm). Now, design an $O(n)$ preprocessing algorithm so that you can answer each query in $O(k)$ time. 
\textbf{Suppose f is the preprocessing function, defined as follows:\\
        1:Use the Median-of-the-Medians algorithm to find the median, denoted as m.\\
        2:Divide the array into three parts: 
        \begin{itemize}
            \item L:those smaller than m;
            \item M:those equal to m;
            \item R:those larger than m.
        \end{itemize}
        As is mentioned above, step 2 takes O(n) time.\\
        Repeat: apply f on L till there is only one integer in L.\\
        After the preprocess above, we can get an array of medians of L, which could be used as a mark in the searching step.\\
        Searching:
        \begin{itemize}
            \item If k < |L|, in that L is an ordered array, T=O(k);
            \item If |L| < k < |L|+|M|, clearly k is the answer;
            \item If k > |L|+|M|, in that $|R| < \frac{n}{2}$, $T=O(\frac{n}{2})<O(k)$.
        \end{itemize}
        In conclusion, the searching step takes O(k) time, which satisfies the conditions.}\\

\miquestion[26]
You are given a 2D discrete topographical map $A[0,\ldots,n-1;0,\ldots,m-1]$ representing a landscape. The number $A[i,j]$ represents the altitude at position $(i,j)$. If it rains over the landscape, the water will form pools at each position where $A[i,j]$ is less than each adjacent position, i.e., those $A[i',j']$ for which $|i-i'|+|j-j'|=1$. (You can assume all altitudes are distinct and that there is a ``wall'' at the edge of the map. For example, water pools at $(0,0)$ if $A[0,0]$ is less than $A[0,1]$ and $A[1,0]$.)
\begin{parts}
    \part Suppose $m=1$, so $A$ is a 1D array. Give a divide and conquer algorithm for finding \emph{one} position where water pools. Write a recurrence for this algorithm. Analyze its running time.\\
    \textbf{Algorithm: Suppose mid=n//2, compare A[1,mid-1],A[1,mid] and A[1,mid+1].\\Apparently, if A[1,mid-1] > A[1,mid] and A[1,mid+1] > A[1,mid], then A[1,mid] is the pool.\\Else, choose the smaller half and repeat the steps above.\\
    Recurrence: T(n)=T(n/2)+O(1).\\ Thus T(n)=O(log n).}
    \part Give another divide and conquer algorithm when $m=n$. Analyze its running time with a recurrence relation.\\
    \textbf{Because m=n, searching direction is still on a line, so actually (2) equals to (1).\\Algorithm: Suppose mid=n//2=m//2, compare A[mid,mid],A[mid+1,mid+1] and A[mid-1,mid-1].\\Apparently, if A[mid,mid] < A[mid+1,mid+1] and A[mid,mid] < A[mid-1,mid-1], then A[mid,mid] is the pool.\\Else, choose the smaller direction and repeat the steps above.\\
    Recurrence: T(n)=T(n/2)+O(1).\\ Thus T(n)=O(log n).}
    \part Generalize your algorithms from part (a) and (b) to work for any $m$ and $n$. The running time should \emph{smoothly} interpolate between the running times of (a) and (b).\\
    \textbf{Algorithm: Suppose $mid_m$=m//2, $mid_n$=n//2, compare A[$mid_m$,$mid_n$] with A[$ mid_m \pm 1 $,$mid_n$] and A[$mid_m$,$mid_n \pm 1$].\\If A[$mid_m$,$mid_n$] is minimum,then it's the pool.\\Else, choose the smallest direction, and repeat the steps above.\\
    Recurrence: for each step, T(m,n)=T(m/2,n)+O(n) or T(m,n/2)+O(m). Thus T(m,n)=O(m+n).}
\end{parts}
  
\miquestion
How long does it take you to finish the assignment (including thinking and discussion)?
Give a score (1,2,3,4,5) to the difficulty.
Do you have any collaborators?
Please write down their names here.\\
 \textbf{About 7 hours.\\
 4\\
 Ruifeng Shang.}

\end{questions}


\end{document}
