\documentclass{article}
\usepackage{amsmath}
\usepackage{amsfonts}
\usepackage{amssymb}
\usepackage{ctex}

\title{Homework 1}
\author{}
\date{}

\begin{document}

\maketitle

\section*{1}
\subsection*{a}
\[
E[Z_t] = E[5 + 2t + X_t] = 5 + 2t + E[X_t] = 5 + 2t
\]

\subsection*{b}
\[
\begin{aligned}
\text{Cov}(Z_t, Z_{t+k}) &= \text{Cov}(5 + 2t + X_t, 5 + 2(t+k) + X_{t+k}) \\
&= \text{Cov}(X_t, X_{t+k}) \\
&= \gamma_k
\end{aligned}
\]

\subsection*{c}
不平稳,因为均值函数5+2t依赖于时间t

\section*{2}
\subsection*{a}
均值:
\[
E(Z_t) = E(a_t) - \theta E(a_{t-1}) = 0 - \theta \sigma_a^2 = -\theta \sigma_a^2
\]
当 $k=0$ 时:

$Cov(Z_t, Z_{t+k}) = Var(Z_t) = Var(a_t) + \theta^2 Var(a_{t-1}) = \sigma_a^2 + 2\theta^2 \sigma_a^4 $


当 $k \neq 0$ 时:

由于 $\{a_t\}$ 独立,
$Z_t$ 和 $Z_{t+k}$ 无重叠项,
\[
\text{Cov}(Z_t, Z_{t+k}) = 0
\]

\subsection*{b}
均值:

\[
E(Z_t) = -\theta \sigma_a^2
\]
为定值。且
\[
\gamma(k) 仅与 k 有关,与时间 t 无关。
\]故 $Z_t$ 是平稳的。

\section*{3}
\subsection*{a}
是

$\{a_t\}$ 独立同分布,均值为0。
方差:
\[
\text{Var}(Z_t) = 2\sigma_a^2 = 2
\]
故ACVF
\[
\text{Cov}(Z_t, Z_{t+k})
\]
仅当 $k=0, \pm2$ 时非零,且仅依赖 $k$。

\subsection*{b}
是

$\{X_t\}$ 平稳,线性变换后均值仍为0。
ACVF 依赖 $\{X_t\}$ 的 ACVF,仅与 $k$ 有关。

\subsection*{c}
否

$\{a_t\}$ 的分布为 $f(x) = 1.5x^{-2.5} \ (x \geq 1)$,由于二阶矩发散,其方差无限大。
故 $Z_t$ 的方差无限。

\subsection*{d}
否

\[
E(Z_t) = 0.5^t
\]
随时间衰减至0,非常数。

\subsection*{e}
是

\[
E(Z_t) = (-1)^t E(X) = 0
\]
方差为
\[
\text{Var}(Z_t) = \text{Var}(X) = 1
\]
故ACVF
\[
\text{Cov}(Z_t, Z_{t+k}) = (-1)^{2t+k} \text{Var}(X) = (-1)^k
\]
仅依赖 $k$。

\subsection*{f}
是

\[
E[(-1)^{Y_t}] = e^{-2}
\]
但 $E(X_t) = 0$,故 $E(Z_t) = 0$。
故ACVF
\[
\text{Cov}(Z_t, Z_{t+k}) = e^{-4} \gamma_k \ (k \neq 0), \ \text{Var}(Z_t) = \gamma_0
\]
仅依赖 $k$,故平稳。

\section*{4}
\subsection*{a}
\[
\begin{aligned}
\gamma_0 &= (1 + 0.25)\sigma_a^2 = 1.25\sigma_a^2 \\
\gamma_1 &= -0.5\sigma_a^2 \\
\rho_1 &=  \frac{-0.5\sigma_a^2}{1.25\sigma_a^2} = -0.4 \\
\rho_k &= 0 \quad (\lvert k \rvert > 1)
\end{aligned}
\]

\subsection*{b}
\[
\begin{aligned}
\gamma_0 &= (1 + 1 + 0.25)\sigma_a^2 = 2.25\sigma_a^2 \\
\gamma_1 &= -1.5\sigma_a^2 \\
\gamma_2 &= 0.5\sigma_a^2 \\
\rho_1  &= \frac{-1.5\sigma_a^2}{2.25\sigma_a^2} = -\frac{2}{3} \\
\rho_2 &=  \frac{0.5\sigma_a^2}{2.25\sigma_a^2} = \frac{2}{9} \\
\rho_k &= 0 \quad (\lvert k \rvert > 2)
\end{aligned}
\]

\subsection*{c}
\[
\begin{aligned}
\gamma_0 &= (1 + 0.25 + 1 + 9)\sigma_a^2 = 11.25\sigma_a^2 \\
\gamma_1 &= -3\sigma_a^2 \\
\gamma_2 &= 0.5\sigma_a^2 \\
\gamma_3 &= 3\sigma_a^2 \\
\rho_1 &=  \frac{-3\sigma_a^2}{11.25\sigma_a^2} = 12/ \\
\rho_2 &=  \frac{0.5\sigma_a^2}{11.25\sigma_a^2} =\frac{2}{45} \\
\rho_3 &=  \frac{3\sigma_a^2}{11.25\sigma_a^2} = \frac{4}{15} \\
\rho_k &= 0 \quad (\lvert k \rvert > 3)
\end{aligned}
\]

\subsection*{d}
\[
\begin{aligned}
Z_t &= (1 - 1.2B + 0.5B^2)a_t \\
\gamma_0 &= (1 + 1.44 + 0.25)\sigma_a^2 = 2.69\sigma_a^2 \\
\gamma_1 &= -1.8\sigma_a^2 \\
\gamma_2 &= 0.5\sigma_a^2 \\
\rho_1 &=  \frac{-1.8\sigma_a^2}{2.69\sigma_a^2} =\frac{180}{269}  \\
\rho_2 &=  \frac{0.5\sigma_a^2}{2.69\sigma_a^2} =\frac{50}{269}  \\
\rho_k &= 0 \quad (\lvert k \rvert > 2)
\end{aligned}
\]

\section*{5}    
特征方程:
\[
r^2 - r - \phi = 0
\]
根为
\[
r = \frac{1 \pm \sqrt{1 + 4\phi}}{2}
\]

复数根:当
\[
1 + 4\phi < 0 \quad (\phi < -\frac{1}{4})
\]
有
\[
|\phi| < 1 \implies \phi > -1
\]

实数根:当
\[
\phi \geq -\frac{1}{4}
\]
要求根绝对值小于1,解得
\[
-1 < \phi < 0
\]

\section*{6}

$\rho_1$ 和 $\rho_2$ 满足 :
\[
\begin{cases}
\rho_1 = \phi_1 + \phi_2 \rho_1 \\
\rho_2 = \phi_1 \rho_1 + \phi_2
\end{cases}
\]
解得:
\[
\rho_1 = \frac{\phi_1}{1 - \phi_2}, \quad \rho_2 = \frac{\phi_1^2}{1 - \phi_2} + \phi_2
\]
代入所需证式,化简得:
\[
|\phi_1| < 1 - \phi_2
\]
又由AR(2)过程的平稳性,有
\[
\begin{cases}
\phi_2 + \phi_1 < 1 \\
\phi_2 - \phi_1 < 1 \\
\end{cases}
\]
故上述条件满足,得证。


\end{document}